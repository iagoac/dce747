\documentclass[compress,mathserif,xcolor=table]{beamer}
\usetheme{sthlm}

%-=-=-=-=-=-=-=-=-=-=-=-=-=-=-=-=-=-=-=-=-=-=-=-=
%        LOADING BEAMER PACKAGES
%-=-=-=-=-=-=-=-=-=-=-=-=-=-=-=-=-=-=-=-=-=-=-=-=

\usepackage{
booktabs,
datetime,
dtk-logos,
graphicx,
multicol,
pgfplots,
ragged2e,
tabularx,
tikz,
wasysym,
multirow,
float,
caption,
subcaption,
amsmath,
mathptmx,
animate
}

\usepackage[scaled=0.9]{helvet}
\usepackage{courier}

\usefonttheme[onlymath]{serif}

\definecolor{mygreen}{RGB}{113, 166, 70}
\definecolor{myblue}{RGB}{68, 140, 185}
\definecolor{myred}{RGB}{217, 98, 55}
\definecolor{mypurple}{RGB}{83, 65, 126}
\definecolor{solviaveis}{RGB}{188, 207, 241}
\definecolor{bronze}{rgb}{0.8, 0.5, 0.2}

\pgfplotsset{compat=1.8}

\usepackage[utf8]{inputenc}
\usepackage[portuguese]{babel}
\usepackage[T1]{fontenc}
\usepackage{newpxtext,newpxmath}
\usepackage{listings}

\lstset{ %
language=[LaTeX]TeX,
basicstyle=\normalsize\ttfamily,
keywordstyle=,
numbers=left,
numberstyle=\tiny\ttfamily,
stepnumber=1,
showspaces=false,
showstringspaces=false,
showtabs=false,
breaklines=true,
frame=tb,
framerule=0.5pt,
tabsize=4,
framexleftmargin=0.5em,
framexrightmargin=0.5em,
xleftmargin=0.5em,
xrightmargin=0.5em
}



%-=-=-=-=-=-=-=-=-=-=-=-=-=-=-=-=-=-=-=-=-=-=-=-=
%        LOADING TIKZ LIBRARIES
%-=-=-=-=-=-=-=-=-=-=-=-=-=-=-=-=-=-=-=-=-=-=-=-=

\usetikzlibrary{
backgrounds,
mindmap
}

%-=-=-=-=-=-=-=-=-=-=-=-=-=-=-=-=-=-=-=-=-=-=-=-=
%        BEAMER OPTIONS
%-=-=-=-=-=-=-=-=-=-=-=-=-=-=-=-=-=-=-=-=-=-=-=-=

\setbeameroption{show notes}

%-=-=-=-=-=-=-=-=-=-=-=-=-=-=-=-=-=-=-=-=-=-=-=-=
%        BEAMER COMMANDS
%-=-=-=-=-=-=-=-=-=-=-=-=-=-=-=-=-=-=-=-=-=-=-=-=


%-=-=-=-=-=-=-=-=-=-=-=-=-=-=-=-=-=-=-=-=-=-=-=-=
%
%	PRESENTATION INFORMATION
%
%-=-=-=-=-=-=-=-=-=-=-=-=-=-=-=-=-=-=-=-=-=-=-=-=

\title{Presente simples, \\ contínuo e perfeito}
\subtitle{DCE747 - Inglês Técnico}
%\date{\small{\jobname}}
\author{\texttt{Iago Carvalho}}
\institute{\texttt{Departamento de Ciência da Computação}}

\hypersetup{
pdfauthor = {Iago A. Carvalho},      
pdfsubject = {Inglês Técnico},
pdfkeywords = {},  
pdfmoddate= {D:\pdfdate},          
pdfcreator = {WriteLaTeX}
}

\begin{document}

\begin{frame}
\titlepage

\end{frame}

%% --------------------------------------------------------

\begin{frame}{Formas de presente em Inglês}

Existem 4 formas de expressar o presente em Inglês
\begin{enumerate}
    \item Simple present (presente simples)
    \item Present continuous (presente contínuo)
    \item Present perfect (Presente perfeito)
    \item Present perfect continuous (Presente perfeito contínuo)
\end{enumerate}

\vspace{0.5cm}

Cada uma das formas conjulga o verbo de maneira diferente
\begin{itemize}
    \item Pode-se também fazer o uso de advérbios
\end{itemize}
\end{frame}

%% --------------------------------------------------------

\begin{frame}{Formas de presente em Inglês}

\textbf{Simple present}: Expressar ações habituais que ocorrem no presente; Explicar sentimentos, desejos, opiniões, fatos, etc.
\begin{itemize}
    \item She watches that show every day
\end{itemize}

\vspace{0.25cm}

\textbf{Present continuous}: Expressar ações que estão ocorrendo no momento da fala; Ações contínuas e que estão ocorrendo
\begin{itemize}
    \item She is watching that show
\end{itemize}

\vspace{0.25cm}

\textbf{Present perfect}: Expressar ações que acabaram de ser finalizadas; Atos que ocorreram no passado e ainda se estendem
\begin{itemize}
    \item She just watched that show
\end{itemize}

\vspace{0.25cm}

\textbf{Present perfect continuous}: Expressar açõe começaram no passado \textit{E} ainda estão ocorrendo no momento da fala
\begin{itemize}
    \item She has been watching that show every day
\end{itemize}

\end{frame}

%% --------------------------------------------------------

\begin{frame}{Simple present}

Ocasionalmente, o simple present também utiliza advérbios de tempo

Os mais comuns são

\begin{minipage}{.49\textwidth}
\begin{itemize}
    \item always (sempre)
    \item every day (todos os dias)
    \item generally (geralmente)
    \item usually (usualmente)
\end{itemize}
\end{minipage}
\begin{minipage}{.49\textwidth}
\begin{itemize}
    \item never (nunca) 
    \item now (agora)
    \item often (frequentemente)
    \item today (hoje)
\end{itemize}
\end{minipage}

\vspace{0.5cm}

Frases no simple present são compostas por dois verbos
\begin{enumerate}
    \item Auxiliar (verbo \textit{to be})
    \item Principal (conjugado no infinitivo)
\end{enumerate}

\end{frame}

%% --------------------------------------------------------

\begin{frame}{Simple present}

\textbf{Forma afirmativa}: Construção mais simples, sem nenhuma modificação

Sujeito + verbo auxiliar (to be) + verbo principal + (...)
\begin{itemize}
    \item He walks on the street. (Ele caminha na rua.)
\end{itemize}

\vspace{0.25cm}

\textbf{Forma negativa}: Adiciona-se o \textit{not} após o \textit{to be}

Sujeito + verbo auxiliar (to be) + not + verbo principal + (...)
\begin{itemize}
    \item He does not walk. (Ele nao caminha.)
\end{itemize}

\vspace{0.25cm}

\textbf{Forma interrogativa}: O verbo \textit{to be} é utilizado no início da frase

Verbo auxiliar (to be) + sujeito + verbo principal + (...)
\begin{itemize}
    \item Does he walks on the street? (Ele caminha na rua?)
\end{itemize}

\end{frame}



%% --------------------------------------------------------

\begin{frame}{Present continuous}

Frases desse tempo verbal se referem ao momento da fala, é bem comum a presença de advérbios de tempo. 

Os mais comuns são

\begin{itemize}
    \item at the moment (no momento)
    \item at the present (atualmente) 
    \item now (agora)
\end{itemize}

\vspace{0.5cm}

Frases no present continuous são compostas por dois verbos
\begin{enumerate}
    \item Auxiliar (verbo \textit{to be})
    \item Principal (conjugado no gerúndio; sufixo -ing)
\end{enumerate}

\end{frame}

%% --------------------------------------------------------

\begin{frame}{Present continuous}

\textbf{Forma afirmativa}: Construção mais simples, sem nenhuma modificação

Sujeito + verbo auxiliar (to be) + verbo principal (com ing) + (...)
\begin{itemize}
    \item He is walking on the street. (Ele está caminhado na rua.)
\end{itemize}

\vspace{0.25cm}

\textbf{Forma negativa}: Adiciona-se o acréscimo do \textit{not} após o \textit{to be}

Sujeito + verbo auxiliar (to be) + not + verbo principal (com ing) + (...)
\begin{itemize}
    \item He is not walking. (Ele nao está caminhado.)
\end{itemize}

\vspace{0.25cm}

\textbf{Forma interrogativa}: O verbo \textit{to be} é utilizado no início da frase

Verbo auxiliar (to be) + sujeito + verbo principal (com ing) + (...)
\begin{itemize}
    \item Is he walking on the street? (Ele está caminhado na rua?)
\end{itemize}

\end{frame}


%% --------------------------------------------------------

\begin{frame}{Present perfect}

Existem alguns advérbios específicos que ajudam a expressar o present perfect

Os mais comuns são

\begin{minipage}{.49\textwidth}
\begin{itemize}
    \item already (já)
    \item always (sempre)
    \item ever (alguma vez)
    \item frequently (frequentemente)
    \item yet (ainda)
\end{itemize}
\end{minipage}
\begin{minipage}{.49\textwidth}
\begin{itemize}
    \item just (já; há pouco tempo) 
    \item lately (ultimamente)
    \item recently (recentemente)
    \item often (frequentemente)
\end{itemize}
\end{minipage}

\vspace{0.5cm}

Frases no simple present são compostas por dois verbos
\begin{enumerate}
    \item Auxiliar (verbo \textit{to have})
    \item Principal (conjugado no \textit{past participle})
\end{enumerate}

\end{frame}

%% --------------------------------------------------------

\begin{frame}{Present perfect}

\textbf{Forma afirmativa}: Construção simples, sem nenhuma modificação

Sujeito + verbo auxiliar (to have) + verbo principal (past participle) + (...)
\begin{itemize}
    \item She has cleaned the house. (Ela limpou a casa.)
\end{itemize}

\vspace{0.25cm}

\textbf{Forma negativa}: Adiciona-se o acréscimo do \textit{not} após o \textit{to be}

Sujeito + verbo auxiliar (to have) + not + verbo principal (past participle) + (...)
\begin{itemize}
    \item She has not cleaned the house. (Ela não limpou a casa.)
\end{itemize}

\vspace{0.25cm}

\textbf{Forma interrogativa}: O verbo \textit{to have} é utilizado no início da frase

Verbo auxiliar (to have) + sujeito + verbo principal (past participle) + (...)
\begin{itemize}
    \item Has she cleaned the house? (Ela limpou a casa?)
\end{itemize}

\end{frame}



%% --------------------------------------------------------

%% --------------------------------------------------------

\begin{frame}{Present perfect continuous}

Misto entre o \textit{present perfect} e o \textit{present continuous}
\begin{itemize}
    \item Expressa ações que se referem ao momento da fala
    \item Ações tiveram início no passado
\end{itemize}

\vspace{0.5cm}

Frases no simple present são compostas por três verbos
\begin{enumerate}
    \item Verbo \textit{to have} no \textit{simple present}
    \item Verbo \textit{to be} no \textit{present perfect}
    \item Principal (conjugado no gerúndio; sufixo -ing)
\end{enumerate}

\end{frame}

%% --------------------------------------------------------

\begin{frame}{Present perfect}

\textbf{Forma afirmativa}: Construção simples, sem nenhuma modificação

Sujeito + verbo \textit{to have} + verbo \textit{to be} + verbo principal (-ing) + (...)
\begin{itemize}
    \item She has been cleaning the house. (Ela tem limpado a casa.)
\end{itemize}

\vspace{0.25cm}

\textbf{Forma negativa}: Adiciona-se o acréscimo do \textit{not} após o \textit{to have}

Sujeito + verbo \textit{to have} + not + verbo \textit{to be} + verbo principal (-ing) + (...)
\begin{itemize}
    \item She has not been cleaning the house. (Ela não tem limpado a casa.)
\end{itemize}

\vspace{0.25cm}

\textbf{Forma interrogativa}: O verbo \textit{to have} é utilizado no início da frase

Verbo to have + sujeito + verbo to be + verbo principal (-ing) + (...)
\begin{itemize}
    \item Has she been cleaning the house? (Ela tem limpado a casa?)
\end{itemize}

\end{frame}

\end{document}