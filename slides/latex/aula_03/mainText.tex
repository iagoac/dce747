\documentclass[compress,mathserif]{beamer}
\usetheme{sthlm}

%-=-=-=-=-=-=-=-=-=-=-=-=-=-=-=-=-=-=-=-=-=-=-=-=
%        LOADING BEAMER PACKAGES
%-=-=-=-=-=-=-=-=-=-=-=-=-=-=-=-=-=-=-=-=-=-=-=-=

\usepackage{
booktabs,
datetime,
dtk-logos,
graphicx,
multicol,
pgfplots,
ragged2e,
tabularx,
tikz,
wasysym,
multirow,
float,
caption,
subcaption,
amsmath,
mathptmx,
animate
}

\usepackage[scaled=0.9]{helvet}
\usepackage{courier}

\usefonttheme[onlymath]{serif}

\definecolor{mygreen}{RGB}{113, 166, 70}
\definecolor{myblue}{RGB}{68, 140, 185}
\definecolor{myred}{RGB}{217, 98, 55}
\definecolor{mypurple}{RGB}{83, 65, 126}
\definecolor{solviaveis}{RGB}{188, 207, 241}

\pgfplotsset{compat=1.8}

\usepackage[utf8]{inputenc}
\usepackage[portuguese]{babel}
\usepackage[T1]{fontenc}
\usepackage{newpxtext,newpxmath}
\usepackage{listings}

\lstset{ %
language=[LaTeX]TeX,
basicstyle=\normalsize\ttfamily,
keywordstyle=,
numbers=left,
numberstyle=\tiny\ttfamily,
stepnumber=1,
showspaces=false,
showstringspaces=false,
showtabs=false,
breaklines=true,
frame=tb,
framerule=0.5pt,
tabsize=4,
framexleftmargin=0.5em,
framexrightmargin=0.5em,
xleftmargin=0.5em,
xrightmargin=0.5em
}



%-=-=-=-=-=-=-=-=-=-=-=-=-=-=-=-=-=-=-=-=-=-=-=-=
%        LOADING TIKZ LIBRARIES
%-=-=-=-=-=-=-=-=-=-=-=-=-=-=-=-=-=-=-=-=-=-=-=-=

\usetikzlibrary{
backgrounds,
mindmap
}

%-=-=-=-=-=-=-=-=-=-=-=-=-=-=-=-=-=-=-=-=-=-=-=-=
%        BEAMER OPTIONS
%-=-=-=-=-=-=-=-=-=-=-=-=-=-=-=-=-=-=-=-=-=-=-=-=

\setbeameroption{show notes}

%-=-=-=-=-=-=-=-=-=-=-=-=-=-=-=-=-=-=-=-=-=-=-=-=
%        BEAMER COMMANDS
%-=-=-=-=-=-=-=-=-=-=-=-=-=-=-=-=-=-=-=-=-=-=-=-=


%-=-=-=-=-=-=-=-=-=-=-=-=-=-=-=-=-=-=-=-=-=-=-=-=
%
%	PRESENTATION INFORMATION
%
%-=-=-=-=-=-=-=-=-=-=-=-=-=-=-=-=-=-=-=-=-=-=-=-=

\title{Estratégias de leitura \\ em Inglês}
\subtitle{DCE747 - Inglês Técnico}
%\date{\small{\jobname}}
\author{\texttt{Iago Carvalho}}
\institute{\texttt{Departamento de Ciência da Computação}}

\hypersetup{
pdfauthor = {Iago A. Carvalho},      
pdfsubject = {Pesquisa Operacional},
pdfkeywords = {},  
pdfmoddate= {D:\pdfdate},          
pdfcreator = {WriteLaTeX}
}

\begin{document}

\begin{frame}
\titlepage

\end{frame}

%% --------------------------------------------------------

\begin{frame}{Existe uma forma simples de ler em inglês?}

A resposta mais simples: \textbf{NÃO!!!}

\vspace{0.5cm}

O que existem são formas de extrair o conhecimento de um texto

\vspace{0.5cm}

Estas estratégias podem ser utilizadas em (quase) qualquer idioma
\begin{itemize}
    \item Inglês
    \item Francês
    \item Italiano
    \item Alemão
    \item $\ldots$
\end{itemize}

\end{frame}

%% --------------------------------------------------------

\begin{frame}{Objetivo da leitura}

A primeira coisa é ter em mente qual é seu objetivo ao ler 
\begin{itemize}
    \item Diferentes textos tem diferentes objetivos
    \item É muito diferente a leitura de um livro de literatura e a de um artigo técnico em computação
\end{itemize}

\vspace{0.5cm}

Pode-se buscar entender o texto de forma ampla

\vspace{0.5cm}

De forma contrária, pode ser que o objetivo seja encontrar uma informação específica dentro do texto

\end{frame}

%% --------------------------------------------------------

\begin{frame}{Identificação de cognatos}

Independente de seu objetivo, a primeira estratégia é simples: a identificação de palavras cognatas!
\begin{itemize}
    \item Também são chamados de palavras transparentes
    \item São palavras em outro idioma que se assemelha a uma em nosso idioma
\end{itemize}

\vspace{0.5cm}

Exemplos de cognatos:
\begin{itemize}
    \item Telephone
    \item Car
    \item History
    \item Garage
    \item Economy
\end{itemize}

\vspace{0.5cm}

A identificação de cognatos nos ajuda a identificar o sentido geral do texto

\end{frame}

%% --------------------------------------------------------

\begin{frame}{Cuidado com os falsos cognatos}

Existem diversas palavras que parecem ser cognatos mas, na verdade, não são!

\vspace{0.5cm}

É de extrema importância saber quais são estas palavras para que o texto seja entendido corretamente

\vspace{0.5cm}

Exemplos de falsos cognatos:
\begin{itemize}
    \item Mayor (prefeito)
    \item Parents (pais)
    \item Pasta (macarrão)
    \item Refrigerant (substância refrigerante)
    \item Push (empurrar)
    \item \href{https://www.sk.com.br/sk-falsos-cognatos-ou-falsos-amigos.html}{\beamergotobutton{Link}}
\end{itemize}
\end{frame}

%% --------------------------------------------------------

\begin{frame}{Palavras repetidas}

Se uma palavra é repetida muitas vezes durante o texto, provavelmente ela é importante!

\vspace{0.5cm}

Vale a pena tirar alguns segundos para descobrir a tradução desta palavra

\vspace{0.5cm}

Entretanto, não é muito comum encontrarmos palavras repetidas diversas vezes
\begin{itemize}
    \item Normalmente, autores utilizam sinônimos de uma mesma palavra para que o texto torne-se menos enfadonho e repetitivo
\end{itemize}

\end{frame}

%% --------------------------------------------------------

\begin{frame}{Pistas tipográficas}

São elementos visuais que ajudam na compreensão do texto

\vspace{0.5cm}

São exemplos de tais elementos visuais
\begin{itemize}
    \item Palavras em \textbf{negrito}
    \item Palavras em \textit{itálico}
    \item \underline{Sublinhado}
    \item LETRAS MAIÚSCULAS
    \item Reticências$\ldots$
    \item "Aspas"
    \item (Parênteses)
\end{itemize}
\end{frame}

%% --------------------------------------------------------

\begin{frame}{\textit{Skimming}}

Esta é uma técnica muito interessante para obtermos a ideia geral do texto
\begin{itemize}
    \item Equivale quase que a dar uma passada rápida de olho
\end{itemize}

\vspace{0.5cm}

A ideia é identificar palavras conhecidas e cognatos presentes no texto
\begin{itemize}
    \item Desta forma, pode-se ter uma ideia geral do que se trata cada parágrafo que você está lendo
    \item Assim, é possível montar uma sequência de raciocínio e entender o sentido global do texto
\end{itemize}

\end{frame}

%% --------------------------------------------------------

\begin{frame}{Objetivos do \textit{Skimming}}

\begin{enumerate}
    \item Obter uma ideia geral do texto: Muito útil na primeira leitura de um texto como forma de construir um \textit{preview} do que está por vir \vspace{0.5cm}
    \item Economia de tempo: Pode-se poupar muito tempo ao dar uma passada de olho no texto ao invés de tentar entende-lo por completo. Isto é muito útil em provas ou testes onde o tempo é um fator essencial
    \vspace{0.5cm}
    \item Revisar um texto já lido anteriormente: quando você já leu e compreendeu um texto, realizar o \textit{skimming}
\end{enumerate}

\end{frame}

%% --------------------------------------------------------

\begin{frame}{Como realizar o \textit{Skimming}}

Leia sempre o início e o fim
\begin{itemize}
    \item Primeira e última frase de um parágrafo
    \item Primeiro e último parágrafo de uma sessão
    \item Primera e última sessão de um texto
    \begin{itemize}
        \item Resumo e conclusões, no caso de artigos científicos
    \end{itemize}
\end{itemize}

Leia somente o necessário para identificar o sentido geral do texto ou parágrafo
\begin{itemize}
    \item A ideia principal aqui é economizar tempo!
\end{itemize}

Atente-se a pistas tipográficas
\begin{itemize}
    \item Normalmente elas contém a ideia central de cada parágrafo
\end{itemize}
\end{frame}

%% --------------------------------------------------------

\begin{frame}{Quando utilizar o \textit{Skimming}}

Ele pode (e deve!) ser utilizado em
\begin{itemize}
    \item Notícias
    \item Textos científicos
    \item Artigos técnicos
\end{itemize}

\vspace{0.5cm}

Entretanto, não é muito útil realizar \textit{skimming} para a compreensão de outros tipos de textos
\begin{itemize}
    \item Textos literários (romances, novelas)
    \item Textos muito curtos
    \item Poesia
\end{itemize}

\end{frame}

%% --------------------------------------------------------

\begin{frame}{Dicas para um bom \textit{skimming}}

\begin{enumerate}
    \item Tenha foco e uma ideia clara do que quer entender
    \item Faça o \textit{skimming} em etapas
    \begin{itemize}
        \item Na primeira, atente-se as figuras, tabelas e títulos
        \item Na segunda, concentre-se na introdução e conclusão
        \item Na terceira, foque individualmente em cada sessão
        \item Na quarta, a ideia é entender cada parágrafo separadamente
    \end{itemize}
    \item Utilize anotações
    \item Resuma as ideias obtidas em parágrafos a parte
    \item Construa questões sobre o texto e tente responde-las
\end{enumerate}

\end{frame}

%% --------------------------------------------------------

\begin{frame}{\textit{Scanning}}

O \textit{scanning} é o inverso do \textit{skimming}

\vspace{0.5cm}

Aqui o objetivo é procurar por ideias e detalhes específicos
\begin{itemize}
    \item Valores
    \item Datas
    \item Alguma palavra específica
\end{itemize}

\vspace{0.5cm}

Esta técnica é muito útil para responder questões de provas ou exames de proeficiência em Inglês

\end{frame}

\end{document}