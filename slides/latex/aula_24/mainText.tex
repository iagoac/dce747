\documentclass[compress,mathserif,xcolor=table]{beamer}
\usetheme{sthlm}

%-=-=-=-=-=-=-=-=-=-=-=-=-=-=-=-=-=-=-=-=-=-=-=-=
%        LOADING BEAMER PACKAGES
%-=-=-=-=-=-=-=-=-=-=-=-=-=-=-=-=-=-=-=-=-=-=-=-=

\usepackage{
booktabs,
datetime,
dtk-logos,
graphicx,
multicol,
pgfplots,
ragged2e,
tabularx,
tikz,
wasysym,
multirow,
float,
caption,
subcaption,
amsmath,
mathptmx,
animate
}

\usepackage[scaled=0.9]{helvet}
\usepackage{courier}

\usefonttheme[onlymath]{serif}

\definecolor{mygreen}{RGB}{113, 166, 70}
\definecolor{myblue}{RGB}{68, 140, 185}
\definecolor{myred}{RGB}{217, 98, 55}
\definecolor{mypurple}{RGB}{83, 65, 126}
\definecolor{solviaveis}{RGB}{188, 207, 241}
\definecolor{bronze}{rgb}{0.8, 0.5, 0.2}

\pgfplotsset{compat=1.8}

\usepackage[utf8]{inputenc}
\usepackage[portuguese]{babel}
\usepackage[T1]{fontenc}
\usepackage{newpxtext,newpxmath}
\usepackage{listings}

\lstset{ %
language=[LaTeX]TeX,
basicstyle=\normalsize\ttfamily,
keywordstyle=,
numbers=left,
numberstyle=\tiny\ttfamily,
stepnumber=1,
showspaces=false,
showstringspaces=false,
showtabs=false,
breaklines=true,
frame=tb,
framerule=0.5pt,
tabsize=4,
framexleftmargin=0.5em,
framexrightmargin=0.5em,
xleftmargin=0.5em,
xrightmargin=0.5em
}



%-=-=-=-=-=-=-=-=-=-=-=-=-=-=-=-=-=-=-=-=-=-=-=-=
%        LOADING TIKZ LIBRARIES
%-=-=-=-=-=-=-=-=-=-=-=-=-=-=-=-=-=-=-=-=-=-=-=-=

\usetikzlibrary{
backgrounds,
mindmap
}

%-=-=-=-=-=-=-=-=-=-=-=-=-=-=-=-=-=-=-=-=-=-=-=-=
%        BEAMER OPTIONS
%-=-=-=-=-=-=-=-=-=-=-=-=-=-=-=-=-=-=-=-=-=-=-=-=

\setbeameroption{show notes}

%-=-=-=-=-=-=-=-=-=-=-=-=-=-=-=-=-=-=-=-=-=-=-=-=
%        BEAMER COMMANDS
%-=-=-=-=-=-=-=-=-=-=-=-=-=-=-=-=-=-=-=-=-=-=-=-=


%-=-=-=-=-=-=-=-=-=-=-=-=-=-=-=-=-=-=-=-=-=-=-=-=
%
%	PRESENTATION INFORMATION
%
%-=-=-=-=-=-=-=-=-=-=-=-=-=-=-=-=-=-=-=-=-=-=-=-=

\title{Coerência, coesão e\\ o uso de conectivos}
\subtitle{DCE747 - Inglês Técnico}
%\date{\small{\jobname}}
\author{\texttt{Iago Carvalho}}
\institute{\texttt{Departamento de Ciência da Computação}}

\hypersetup{
pdfauthor = {Iago A. Carvalho},      
pdfsubject = {Inglês Técnico},
pdfkeywords = {},  
pdfmoddate= {D:\pdfdate},          
pdfcreator = {WriteLaTeX}
}

\begin{document}

\begin{frame}
\titlepage

\end{frame}

%% --------------------------------------------------------

\begin{frame}{Conectivos}

Linking words (conectivos) em Inglês são palavras que tem como função estabelecer a ligação entre duas frases.
\begin{itemize}
    \item Função similar as conjunções da língua portuguesa
    \item São palavras invariáveis
    \begin{itemize}
        \item Não sofrem flexão de grau, número ou gênero
    \end{itemize}
\end{itemize}

\vspace{0.5cm}

Desta forma, o correto uso de conectivos é muito importante para um texto em Inglês

\vspace{0.5cm}

Conectivos dão coerência e coesão a um texto
\begin{itemize}
    \item Organiza e interliga diversas orações
    \item A correta utilização de conectivos dá uma sequência lógica as ideias expressas no texto
\end{itemize}

\end{frame}

%% --------------------------------------------------------

\begin{frame}{Diferentes classes de conectivos}

Existem diversas classes de conectivos

\vspace{0.25cm}

\begin{minipage}{.49\textwidth}
\begin{itemize}
    \item Introdução e continuação
    \item Adição e similaridade
    \item Alternância ou exclusão
    \item Causa e explicação
    \item Consequência
\end{itemize}
\end{minipage}
\begin{minipage}{.49\textwidth}
\begin{itemize}
    \item Contraste
    \item Condição
    \item Exemplificação
    \item Ênfase
    \item Conclusão
\end{itemize}
\end{minipage}

\vspace{1cm}

Conhecer os conectivos e suas funções ajuda na compreensão dos textos
\end{frame}

%% --------------------------------------------------------

\begin{frame}{Introducão e continuação}

\textbf{Introdução}: Iniciar frases ou parágrafos

\vspace{0.25cm}

\begin{minipage}{.49\textwidth}
\begin{itemize}
    \item First of all (Antes de tudo)
    \item To begin with (Para começar)
\end{itemize}
\end{minipage}
\begin{minipage}{.49\textwidth}
\begin{itemize}
    \item In the first place (Em primeiro lugar)
    \item First (Primeiro)
\end{itemize}
\end{minipage}

\vspace{0.25cm}

\textbf{Continuação}: Continuidade de um conceito ou ideia

\vspace{0.25cm}

\begin{minipage}{.49\textwidth}
\begin{itemize}
    \item Besides that (Além disso)
    \item Furthermore (Ademais)
\end{itemize}
\end{minipage}
\begin{minipage}{.49\textwidth}
\begin{itemize}
    \item Moreover (Além do mais)
    \item Then (Depois, em seguida)
\end{itemize}
\end{minipage}

\vspace{0.5cm}

\textbf{To begin with}, we never did anything close to it! \textbf{Furthermore}, we were home yesterday. \\
\vspace{0.15cm}
\textbf{Para começar}, nós nunca fizemos algo parecido! \textbf{Além disso}, nós estávamos em casa ontem.
\end{frame}



%% --------------------------------------------------------

\begin{frame}{Adição ou similaridade}

\textbf{Adição e similaridade}: Adicionar ideias, apresentar novas informações ao que já foi exposto

\vspace{0.25cm}

\begin{minipage}{.49\textwidth}
\begin{itemize}
    \item And (E)
    \item Also (Além disso)
\end{itemize}
\end{minipage}
\begin{minipage}{.49\textwidth}
\begin{itemize}
    \item As well as (Assim como)
    \item In addition to (Além de)
\end{itemize}
\end{minipage}

\vspace{0.5cm}

I asked for a cake \textbf{and} a coffee. \textbf{Also}, I bought a juice \\
\vspace{0.15cm}
Eu pedi um bolo \textbf{e} um café. \textbf{Além disso}, eu comprei um suco

\vspace{0.25cm}

Funk, \textbf{as well as} country music, is a very popular kind of music in Brazil. \\
\vspace{0.15cm}
Funk, \textbf{assim como} sertanejo, é um estilo musical muito popular no Brasil.
\end{frame}

%% --------------------------------------------------------

\begin{frame}{Alternância e exclusão}

\textbf{Alternância}: Conectivo \textbf{or} (ou)

\vspace{0.15cm}

Do you prefer coke or pepsi? \\
Você prefere coca ou pepsi?

\vspace{0.15cm}

\textbf{Exclusão}: Apresenta duas ou mais opções, excluindo uma ou mais delas

\vspace{0.1cm}

Neither ... nor ... (Nem ... nem ...)
\begin{itemize}
    \item Liga duas ideias negativas
\end{itemize}
Either ... or ... (ou ...)
\begin{itemize}
    \item Aponta duas possibilidades
\end{itemize}

\textbf{Neither} my girlfriend \textbf{nor} my brother can cook. \\
\textbf{Nem} minha namorado \textbf{nem} meu irmão sabem cozinhar. 

I can travel \textbf{either} by car \textbf{or} by bus
 \\
Eu posso viajar de carro \textbf{ou} de ônibus
\end{frame}

%% --------------------------------------------------------

\begin{frame}{Causa e explicação}

\textbf{Causa e explicação}: Utilizados para explicar uma informação dada anteriormente; Para exibir as causas de uma ação

\vspace{0.25cm}

\begin{minipage}{.49\textwidth}
\begin{itemize}
    \item As (Como)
    \item Because (Porque)
\end{itemize}
\end{minipage}
\begin{minipage}{.49\textwidth}
\begin{itemize}
    \item For (Para, porquê)
    \item Since (Desde que, já que)
\end{itemize}
\end{minipage}

\vspace{0.5cm}

The party flopped \textbf{because} the music was horrible. \\
\vspace{0.15cm}
A festa fracassou \textbf{porquê} a música estava horrível.

\vspace{0.25cm}

I bought this car \textbf{for} work \textbf{since} the old one was robbed \\
\vspace{0.15cm}
Eu comprei este carro \textbf{para} trabalhar \textbf{já que} o antigo foi roubado

\end{frame}



%% --------------------------------------------------------

\begin{frame}{Consequência e constraste}

\textbf{Consequência}: Iniciar as consequências de uma ação anterior

\vspace{0.25cm}

\begin{minipage}{.49\textwidth}
\begin{itemize}
    \item So (Então)
    \item Thus (Deste modo, assim)
\end{itemize}
\end{minipage}
\begin{minipage}{.49\textwidth}
\begin{itemize}
    \item Therefore (Deste modo)
    \item Since (Desde que)
\end{itemize}
\end{minipage}

\vspace{0.25cm}

\textbf{Contraste}: Indicar a existência de conceitos ou ideias opostos a algo apresentado

\vspace{0.25cm}

\begin{minipage}{.49\textwidth}
\begin{itemize}
    \item However (Entretanto)
    \item But (Mas)
    \item Otherwise (Caso contrário)
\end{itemize}
\end{minipage}
\begin{minipage}{.49\textwidth}
\begin{itemize}
    \item Although (Apesar de, embora)
    \item Instead of (Ao invés de)
\end{itemize}
\end{minipage}

\vspace{0.5cm}

I studied a lot. \textbf{Thus}, I got a good grade. \textbf{However}, my brother drank all weekend, \textbf{so} he got a zero. \\
\vspace{0.15cm}
Eu estudei mujito. \textbf{Assim}, eu consegui uma boa nota. \textbf{Entretanto}, meu irmão bebeu todo o final de semana, \textbf{então} ele tirou um zero.

\end{frame}

%% --------------------------------------------------------

\begin{frame}{Condição e exemplificação}

\textbf{Condição}: Para dizer que algo só sera realizado mediante uma condição previamente estabelecida

\vspace{0.15cm}

\begin{minipage}{.49\textwidth}
\begin{itemize}
    \item If (Se)
    \item Unless (A não ser que)
\end{itemize}
\end{minipage}
\begin{minipage}{.49\textwidth}
\begin{itemize}
    \item As long as (Desde que)
\end{itemize}
\end{minipage}

\vspace{0.25cm}

\textbf{Exemplificação}: Fornecer exemplos para ilustrar ideias ou conceitos

\vspace{0.15cm}

\begin{minipage}{.49\textwidth}
\begin{itemize}
    \item For example (Por exemplo)
    \item For instance (Por exemplo)
\end{itemize}
\end{minipage}
\begin{minipage}{.49\textwidth}
\begin{itemize}
    \item In this case (Neste caso)
    \item Such as (Tal como)
\end{itemize}
\end{minipage}

\vspace{0.25cm}

\textbf{If} you work out everyday, you will be stronger, \textbf{unless} if you have a bad diet. \textbf{For example}, eating a lot of sugar and fat will minimize the results of your exercises. \\
\vspace{0.15cm}
\textbf{Se} você treinar todos os dias, você ficará mais forte, \textbf{a não ser que} você tenha uma dieta ruim. \textbf{Por exemplo}, comer um monte de açucar e gordura vai minimizar os resultados de seus exercícios.

\end{frame}

%% --------------------------------------------------------

\begin{frame}{Ênfase e conclusão}

\textbf{Ênfase}: Destacar informações expostas

\vspace{0.15cm}

\begin{minipage}{.49\textwidth}
\begin{itemize}
    \item Above all (Sobretudo)
    \item Indeed (De fato, realmente)
\end{itemize}
\end{minipage}
\begin{minipage}{.49\textwidth}
\begin{itemize}
    \item Even more (Ainda mais)
    \item Most of all (Acima de tudo)
\end{itemize}
\end{minipage}

\vspace{0.5cm}

\textbf{Conclusão}: Indicar o fim de um conceito ou explicação

\vspace{0.15cm}

\begin{minipage}{.49\textwidth}
\begin{itemize}
    \item After all (Afinal)
    \item At last (Por último)
\end{itemize}
\end{minipage}
\begin{minipage}{.49\textwidth}
\begin{itemize}
    \item Finally (Finalmente)
    \item To summarize (Resumindo)
\end{itemize}
\end{minipage}

\vspace{0.5cm}

\textbf{To summarize}, we worked a lot. \textbf{Most of all}, our team, \textbf{indeed}, did an excellent job in this project.
 \\
\vspace{0.15cm}
\textbf{Para resumir}, nós trabalhos muito. \textbf{Acima de tudo}, nosso equipe, \textbf{de fato}, realizou um excelente trabalho neste projeto.

\end{frame}
\end{document}